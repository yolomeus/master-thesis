\documentclass[headsepline]{scrreprt}

\usepackage[utf8]{inputenc}
\usepackage{csquotes}
\usepackage[english]{babel}

\usepackage[round]{natbib}
\bibliographystyle{abbrvnat}

\usepackage{scrlayer-scrpage}
\pagestyle{scrheadings}
\automark[section]{section}
\clearpairofpagestyles
\ohead{\headmark}
\ofoot[\pagemark]{\pagemark}

\usepackage{subcaption}

\usepackage[]{graphicx}
\graphicspath{{gfx/}}
\DeclareGraphicsExtensions{.png,.jpg,.pdf}

\usepackage{amsfonts}
\usepackage{amsmath}
\usepackage{stmaryrd}
\usepackage{hyperref}
\usepackage{cleveref}

\usepackage[ruled,vlined]{algorithm2e}
\usepackage{todonotes}

% CUSTOM COMMANDS
\newcommand{\handindate}{10.08.2022}
\newcommand{\ti}[1]{\textit{#1}}
\newcommand{\tx}[1]{\text{#1}}
\newcommand{\tf}[1]{\textbf{#1}}
\newcommand{\R}{\mathbb{R}}
\newcommand{\MSubject}{\mathcal{M}}
\newcommand{\lay}[1]{^{(#1)}}

\begin{document}
\thispagestyle{empty}

{\sffamily
	\begin{figure}
		\includegraphics[height=20mm]{logo_luh}
		\hfill
		\includegraphics[height=20mm]{l3s_logo}
	\end{figure}
	\begin{center}
		\mbox{} \\
		\vspace{1.5cm}
		\huge{Deconstructing Ranking Abilities of

			Language Models}\\
		\vspace{1.8cm}
		{\large
			Gottfried Wilhelm Leibniz Universität Hannover\\
			Fakultät für Elektrotechnik und Informatik\\
			Institut für verteilte Systeme \\
			Fachgebiet Wissensbasierte Systeme \\
			Forschungszentrum L3S \\
			\vspace{1.8cm}
			Thesis by \\
			\textbf{Fabian Beringer} \\
			\vspace{1.8cm}
			\begin{tabular}{rl}
				First examiner:  & Prof. Dr. Wolfgang Nejdl \\
				Second examiner: & Dr. Sowmya S. Sundaram   \\
				Advisor:         & Jonas Wallat             \\
			\end{tabular}
			\vfill
			Hannover, \handindate
		}
	\end{center}
}
\newpage

\chapter*{Abstract}
Nowadays, information retrieval plays an important role in our daily lives. Whether we're searching the web, shopping for products online, or trying to find our favorite movies on a streaming platform: An information retrieval system will be responsible for tackling these tasks. As a consequence of recent advances in natural language processing, employing large pre-trained language models as part of a text retrieval pipeline has become a common approach\footnote{\url{https://searchengineland.com/google-bert-used-on-almost-every-english-query-342193}}. However, despite their proven effectiveness, these neural network based models are functional black boxes, meaning it is not clear to us as to how they arrive at certain decisions. To get a better understanding of the inner workings of such a model, we apply the recently emerging \ti{probing} paradigm. By employing a diagnostic classifier, this approach enables us to analyze how certain properties are encoded within a model's hidden representations. Unlike previous research that has focused on general linguistic properties, we explicitly study the layer-wise distribution of ranking related knowledge throughout the popular BERT model, a large neural network that has been trained on massive amounts of text data. In this thesis, we provide evidence that BERT not only stores ranking related concepts, but also orders them in a hierarchical manner. Moreover, we leverage our findings to design a multi-task learning setup which infuses task specific information at different layers of BERT, in order to improve the model's ability to rank.

\tableofcontents

\chapter{Introduction}

\section{Motivation}
Nowadays, modern approaches for NLP often rely on large language models. These models are first pre-trained on huge amounts of text data and then fine-tuned on a smaller, task specific dataset. Although this approach has shown great effectiveness compared to traditional approaches, due to their size and complexity, these language models are mostly being treated as black boxes.

This is where several issues arise.

goal
shine more light on how
probing specifically targeted to IR
use knowledge to improve



\section{Problem Statement}
I am still alive.
- shine light on knowledge dist
- research question
\section{Contribution}

\section{Thesis Outline}

\chapter{Foundations}
\label{foundations}
\section{Information Retrieval - Ranking Text}
Ranking is an integral part of the information retrieval (IR) process. The general IR problem can be formulated as follows: A user with a need for information expresses this information need through formulation of a query. Now given the query and a collection of documents, the IR system's task is to provide a list of documents that satisfy the user's information need. Further, the retrieved documents should be sorted by relevance w.r.t. the user's information need in descending order, i.e. the documents considered most relevant should be at the top of the list.

While from this formulation only, the task might appear simple, there are several caveats to look out for when it comes to ranking. For instance, there is no restriction on the structure of the query. While we might expect a natural language question like "What color are giraffes?" a user might decide to enter a keyword query like "giraffes color". The same applies to documents: Depending on the corpus we are dealing with, the documents might be raw text, structured text like HTML or even another type of media e.g. image, audio or a combination thereof.

Another possible issue is a mismatch in information need of the user and the corresponding query. Even if we find a perfect ordering of documents with respect to the query, we can not know for certain that the query actually reflects the user's information need. The user might not even know exactly what they're looking for until discovery through an iterative process, i.e. the information need is fuzzy and can not be specified through an exact query from the beginning on.

Further, a query might require additional context information in order for an IR system to find relevant documents. For example, depending on the time at which a query is prompted, the correct answer might change: "Who is president?" should return a different set of documents, as soon as a new president has been elected. Also, since not specified further, it is up to interpretation which country's president the user is interested in and might depend on their location.
In addition, even the corpus might not be static either and change or grow over time, e.g. web search has to deal with an ever-growing corpus: the internet.

While this list of issues is not comprehensive, at this point the complexity of the ranking problem should have become apparent.

Because this work focuses on the ranking of text in the context of web search, we will now give a formal definition with that scenario in mind:

Given a set of $|Q|$ natural language queries $Q = \{q_i\}_{i=1}^{|Q|}$ and a collection of $|D|$ documents $D = \{d_i\}_{i=1}^{|D|}$, we want to find a scoring function $s: Q \times D \rightarrow \R$, such that for any query $q \in Q$ and documents $d, d' \in D$, it holds true that $s(q, d) > s(q, d')$ if $d$ is more relevant w.r.t $q$ than $d'$.

To give the reader a more concrete idea and as we are going to build upon it throughout this work, we will now discuss two traditional approaches to text retrieval which, unlike neural retrieval, are based on exact matching, meaning query and document terms are compared directly. Further, they're "bag of word" models, meaning queries and documents are treated as sets of terms without considering order.

\subsection{TF-IDF}
Term Frequency - Inverted Document Frequency weighting (TF-IDF), is a traditional ranking approach that, given a query, assigns a relevance score to each document based on two assumptions:
\begin{enumerate}
    \item A document is relevant if terms from the query appear in it often.
    \item A document is relevant if the terms shared with the query are also rare in the collection.
\end{enumerate}

From these assumptions, two metrics are derived:
\begin{enumerate}
    \item Term-Frequency
          \begin{equation}
              w_{t,d} = \begin{cases}
                  1 + \log \tx{tf}_{t, d} & \tx{if } \tx{tf}_{t, d} > 0 \\
                  0                       & \tx{otherwise}              \\
              \end{cases}
          \end{equation}

          where $\tx{tf}_{t, d}$ is the count of term $t$ in document $d$. The logarithmic scaling is motivated by the idea that a document does not linearly become more relevant by the number of terms in it: A document containing a term 10 times more often doesn't necessarily mean it is $10$ times more important, e.g. the document might just be very long and contain more words in general. Note that this is just one possible normalization scheme out of many.

    \item Inverted Document Frequency
          \begin{equation}
              \tx{idf}(t, d) = \log \frac{|D|}{\tx{df}_t}
          \end{equation}
          where $\tx{df}_t$ counts the number of documents that a term occurs in over the full corpus. This way, terms that occur less frequent relative to the corpus size will receive a high IDF score and those that are more frequent a lower score.

\end{enumerate}

\begin{table}

\end{table}
To compute TF-IDF we can simply sum over the product of TF and IDF for each term in the query to produce a relevance score:
\begin{equation}
    \tx{score}(q, d) = \sum_{t \in q} w_{t,d} \times \tx{idf}_t
\end{equation}

Alternatively, vector space idf vector
\subsection{BM25}

\section{Machine Learning}
Machine learning can be described as a set of statistical methods, for automatically recognizing and extracting patterns from data. Typically, we can distinguish between two main types of machine learning: Supervised learning and unsupervised learning.

In the case of supervised learning, we have a set of training instances $X = \{x_i\}_{i=1}^N$ and corresponding labels $Y = \{y_i\}_{i=1}^N$, assigning a certain characteristic to each data point. For example, this characteristic might be a probability distribution over a set of classes or a regression score. If each $y_i$ represents one or more categories from a fixed set of classes $C = \{c_i\}_{i=1}^{|C|}$, this is called a classification problem.

Now given the training data and labels, the goal is to find a hypothesis that explains the data, such that for unseen data points $x' \notin X$, the labels $y' \notin Y$ can be inferred automatically. One way to estimate the generalization ability of a model or algorithm, is to divide the dataset into a training and a test set, and only train on the training set while using the test set for evaluation. If the test set models the full distribution of data adequately, it can act as a proxy for estimating the error on unseen samples.

In contrast, in unsupervised learning there is no access to any labels whatsoever. Characteristics of the data need to be learned solely from the data $X$ itself. Examples for this include clustering where $X$ is clustered into groups, representation learning which usually tries to find vector representations for $X$, as well as dimensionality reduction that, if each $X$ is already a vector, tries to compress them into more compact but still informative representations.

That being said, the separating lines between supervised and unsupervised learning are blurry. Especially with the emergence of semi-supervised approaches and "end2end" representation learning, modern ML methods often integrate parts of both.

\section{Deep Learning}
Deep learning is a subfield of ML that makes use of a class of models called Deep Neural Networks (DNN). Typically, DNNs find application in the supervised learning scenario and are often used for classification tasks. In the following we explain the basic mechanisms of DNNs and common approaches to train them.

\subsection{Deep Neural Networks}
In essence, a Deep Neural Network (DNN) is a function approximator $f: \R^n \rightarrow \R^m$ that applies a series of non-linear transformations to its inputs, in order to produce an output. In its simplest form, an input vector $x \in \R^n$ is multiplied by a single weight matrix, a bias vector is added, and the resulting vector is passed through a non-linear activation function $\sigma$.

\begin{equation}
    f(x) = \sigma(W x + b)
\end{equation}

where $W \in \R^{m \times n}$ and $b \in \R^{m}$ are learnable parameters.
This model is commonly referred to as single layer feed-forward neural network (FFN) or single layer perceptron.

When used for classification, a single layer FFN is limited to problems that require linear separation. In order to learn more complex, non-linear decision boundaries, multiple layers can be applied in sequence.

An $L$-layer DNN can be described as follows:

\begin{equation}
    \label{eq:DNN}
    \begin{split}
        h^{(1)} &= \sigma^{(1)}(W^{(1)} x + b^{(1)}) \\
        h^{(2)} &= \sigma^{(2)}(W^{(2)} h^{(1)} + b^{(2)}) \\
        & \vdots \\
        f(x) &= \sigma^{(L)}(W^{(L)} h^{(L-1)} + b^{(L)})
    \end{split}
\end{equation}

\todo{popular activations}


\subsection{Optimization}
Arguably, the most common way for optimizing a neural network are the gradient descent (GD) algorithm and its variants. For this, an objective function $J(\theta)$ is defined, based on the DNN's outputs and corresponding target labels over the training set.

\begin{equation}
    J(\theta) = \frac{1}{N} \sum_{i=1}^{N} \mathcal{L}(f(x_i; \theta), y_i)
\end{equation}

Here, $\mathcal{L}$ is a differentiable loss function and $\theta$ represents the vector of all learnable parameters of the neural network $f(x)$.

\subsubsection{Gradient Descent}
For GD, the gradient of $J(\theta)$ with respect to $\theta$ is computed and scaled by a hyperparameter called learning rate $\eta$. If the objective is to minimize, the scaled gradient is subtracted from the original parameter vector.

\begin{equation}
    \theta_{new} = \theta - \eta\nabla_\theta J(\theta)
\end{equation}

By repeating this procedure iteratively, we can gradually minimize $J(\theta)$.

Common choices for $\mathcal{L}$ include:
\begin{itemize}
    \item \textbf{Cross Entropy Loss}
          \begin{equation}
              \tx{CE}(y, \hat{y}) = - \sum_{k=1}^C y_k \log \hat{y}_k
          \end{equation}
          for classification tasks. Where $y_k$ is the ground truth probability of class $k$ and $\hat{y}_k$ the corresponding prediction.

    \item \textbf{Mean Squared Error}
          \begin{equation}
              \tx{MSE}(y, \hat{y}) = (y - \hat{y})^2
          \end{equation}
          in the case of regression.
\end{itemize}

\subsubsection{Stochastic Gradient Descent}
The aforementioned algorithm is also known as the batch gradient descent (BGD) variant. Stochastic Gradient Descent (SGD) differs from BGD in the number of training samples that are used for a gradient update. Where BGD uses the gradient of the full training set for updating $\theta$, SGD only considers a single, randomly picked sample for each update. Not only can this approach be more efficient, since less redundant computations are performed, due to its stochastic nature and high variance, it is more likely to break out of local minima, allowing additional exploration for better solutions. \cite{ruder2016overview}

\subsubsection{Mini-Batch Gradient Descent}
While SGD's high variance during training makes it more likely to escape local minima, it can also come with the disadvantage of unstable training. In this scenario, convergence might be hindered by overshooting desirable minima.

To mitigate this issue, we can simply use more than $1$ sample, in order to achieve a more accurate estimate of the full gradient. Now, at each step a small subset of the dataset is sampled to reduce variance and stabilize training while retaining a level of stochasticity. This variant of gradient descent is called mini-batch gradient descent.

Building on mini-batch GD, many algorithms have been introduced in the context of DNNs, that employ further optimizations in order to improve convergence speed and quality. Notable examples include:
\begin{itemize}
    \item Adagrad \cite{duchi2011adaptive}
    \item RMSProp \cite{hinton2012neural}
    \item Adam \cite{kingma2014adam}
\end{itemize}

\begin{algorithm}
    \SetAlgoLined
    \KwData{$X=\{(x_0, y_0), ..., (x_n, y_n)\}$ training examples and target labels.}
    \KwIn{function $f$ with trainable parameters $\theta$}
    initialize $\theta$ with random values \;
    \While{not converged}{
    $B \leftarrow \tx{next k training pairs} \in X$ \;
    $\theta \leftarrow \theta - \eta\nabla_\theta\big(\frac{1}{k} \sum_{(x_i, y_i) \in B}\mathcal{L}(f(x_i; \theta), y_i)\big)$ \;
    }
    \caption{Mini-Batch Gradient Descent with batch size $k$, learning rate $\eta$}
\end{algorithm}

\subsubsection{Backpropagation}
\label{sec:backprop}
Because a neural network can consist of multiple layers and thus, is a composition of multiple non-linear functions, computing the gradient w.r.t. to each parameter of the network can become a non-trivial and even cumbersome task, if done by hand. One popular way of automatically computing the gradients of a DNN is the backpropagation algorithm \cite{rumelhart1988learning}.

Backpropagation is a direct application of the chain rule for calculating the derivative of the composition of two functions. Given two differentiable functions $f(x)$ and $g(x)$, the chain rule states that the derivate of their composition $f(g(x))$ is equal to the partial derivative of $f$ w.r.t. $g$, times the partial derivate of $g$ w.r.t $x$.

\begin{equation}
    \frac{\partial f(g(x))}{\partial x} = \frac{\partial f(g(x))}{\partial g(x)} \frac{\partial g(x)}{\partial x}
\end{equation}

Let $a^{(k)} = W^{(k)} h^{(k-1)} + b^{(k)}$ be the intermediate output of an $L$-layer DNN at layer $k$, before passing it through an activation function $\sigma$ (See \ref{eq:DNN}). With a single application of the chain rule, we can compute the gradient of the objective function $J$ w.r.t. $a^{(L)}$ like so:

\begin{equation}
    \frac{\partial J}{\partial a^{(L)}} = \frac{\partial J}{\partial \sigma(a^{(L)})} \frac{\partial \sigma(a^{(L)})}{\partial a^{(L)}}
\end{equation}

If we now apply the chain rule a second time, we can produce a term for computing the derivative w.r.t. $W^{(L)}$.

\begin{equation}
    \frac{\partial J}{\partial W^{(L)}} = \frac{\partial J}{\partial \sigma(a^{(L)})} \frac{\partial \sigma(a^{(L)})}{\partial a^{(L)}} \frac{\partial a^{(L)}}{W^{(L)}}
\end{equation}

Note that we now only need to know the derivatives of $J$, $\sigma$ and $a^{(L)}$ separately, in order to compute the derivative of their composition. By recursively applying this rule, we can compute partial derivatives of $J$ w.r.t to parameters of the DNN, up to an arbitrary depth, as long as all functions it is composed of are differentiable.

By modeling the chain of operations in a DNN as a computation graph, deep learning frameworks like PyTorch \cite{NEURIPS2019_9015} or Tensorflow \cite{tensorflow2015-whitepaper} can automatically perform backpropagation, as long as each operation's derivative is known and pre-defined in the library.

\subsection{Regularization}
Regularization includes a number of techniques to improve the generalization capabilities of an ML model. If an ML model achieves a low error rate on training data, but a high error rate on test data, it is said to be overfitting. In this scenario, the model has essentially "memorized" the training data and can no longer adapt to unseen examples. Regularization techniques tackle this problem by limiting the hypothesis space of models, through favoring simple solutions over complex ones.

\subsubsection{Weight Decay}
Weight decay constraints the number of possible hypothesis, by adding a penalty based on model parameters. For example, L2-regularization encourages small weights that lie on a hypersphere, by adding the sum of squares over all parameters to the loss function.

\begin{equation}
    J(\theta) = \mathcal{L}(\theta) + \lambda ||\theta||_2^2
\end{equation}

As L2 is only a soft constraint, its effect can be regulated by hyperparameter $\lambda$.

\subsubsection{Dropout}
Dropout is a DNN specific method that, during training time, randomly sets entries in the input vector of a layer to $0$ with probability $p$ \cite{DBLP:journals/corr/abs-1207-0580}. The initial idea of this approach is, to prevent groups of neurons from co-adapting, i.e. requiring the activation of one another in order to detect a certain feature. If dropout is employed, a neuron can no longer rely on the presence of another neuron. Dropout can also be seen as a way of training an ensemble of sub-networks of the original network which share the same parameters.


\section{Transformer Models}
One of the most prominent deep learning architectures of the past years is the transformer \cite{vaswani2017attention}. The transformer and its variants have set multiple state-of-the-art records in a variety of NLP tasks \cite{devlin-etal-2019-bert, DBLP:journals/corr/abs-1907-11692, DBLP:journals/corr/abs-2003-10555, DBLP:journals/corr/abs-1909-08053, DBLP:journals/corr/abs-2005-14165}, and have since then also been adapted to other domains such as computer vision \cite{DBLP:journals/corr/abs-2010-11929} or audio generation \cite{https://doi.org/10.48550/arxiv.2005.00341}.

In this section we will discuss the architecture and ideas behind it and explain one of the most popular training approaches for NLP, named BERT \cite{devlin-etal-2019-bert}.

\subsection{Architecture}
The transformer architecture is based on a single building block which, after an input layer, is repeatedly applied in order to form the full model. Throughout this thesis we will also refer to these building blocks as layers, e.g. a $12$-layer model consists of an input layer followed by $12$ blocks. A single transformer block consists of:
\begin{itemize}
    \item a multi-head attention layer
    \item a point-wise feed-forward layer
    \item residual connections
    \item layer normalization
\end{itemize}

We will first explain the input layer, then go over each of these elements and explain how a transformer block is constructed from them.

\subsection{Input Layer}
The transformer's input layer takes in a sequence of tokens and produces continuous representations, by selecting corresponding vectors from a learned embedding matrix $M \in \R^{d \times |V|}$. Here $|V|$ denotes the size of the vocabulary and $d$ the hidden dimension of the model. Because the transformer model itself does not encode the order of inputs, \ti{positional encodings} are added to the initial embeddings, in order to inject positional information.

One way for generating positional encodings, is to introduce a new set of learned embeddings of dimension $d$, one for each input position. However, this approach requires a fixed maximum input length, as all embeddings have to be defined before training. Alternatively, \cite{vaswani2017attention} propose to use sine and cosine waves, as a function of the position:

\begin{equation}
    \text{PE}_{(pos, 2i)} = \sin\bigg(\frac{pos}{10000^{\frac{2i}{d}}}\bigg)
\end{equation}
\begin{equation}
    \text{PE}_{(pos, 2i + 1)} = \cos\bigg(\frac{pos}{10000^{\frac{2i}{d}}}\bigg)
\end{equation}

Where $pos$ and $i$ denote position along in along sequence and hidden dimension respectively. They found this approach to perform nearly identical to learned embeddings in the case of machine translation.

\subsection{Multi-Head Self-Attention}
\subsubsection{The Attention Mechanism}
Originally, attention mechanisms have been proposed in the context of neural machine translation (NMT) \cite{bahdanau2014neural,luong2015effective}. Particularly, they were used for aligning words from a source language with their corresponding translations, i.e. pointing out the source words that are relevant for predicting the next translation target. The alignment is important, as different languages usually do not share the same word order, making a sequential word-to-word translation infeasible.

Generally speaking, attention is a mechanism that allows a model to focus on parts of its inputs, usually while considering a certain context. This could for example be words in a text (inputs) that are regarded important for answering a question (context) \cite{xiong2016dynamic} or patches of pixels in an image (inputs) that are important for detecting a certain object type (context) \cite{xu2015show}.

Given a sequence of $N$ input vectors $X = (x_1, \dots, x_N) \in \R^{N \times d}$ and $M$ context vectors $C = (c_1, \dots, c_M) \in \R^{M \times d}$, we can describe the general attention mechanism as follows:

\begin{align}
     & s_{ij} = a(x_i, c_j)                                         \\
     & \alpha_{ij} = \frac{\exp(s_{ij})}{\sum_{k=1}^N \exp(s_{kj})} \\
     & h_j = \sum_{k=1}^N \alpha_{kj} x_k
\end{align}

Where $a: \R^d \times \R^d \rightarrow \R$ is a scoring function that assigns importance scores to input $x_i$ given context $c_j$. The attention weights $\alpha_{ij}$ are then used to produce a context-sensitive representation $h_j$ as weighted sum of $X$.

Common choices for $a$ include:
\begin{itemize}
    \item $a(u, v) = u \cdot v$ (dot product)
    \item $a(u, v) = w^\top \tanh(W u + U v)$ (additive)
    \item $a(u, v) = \sigma(w^\top \tanh(W u + U v + b) + c)$ (MLP)
\end{itemize}

\subsubsection{Self-Attention}
Self-Attention is a special case of attention where input vectors and context vectors stem from the same input sequence. It can be seen as the model attending to a sequence, given the sequence itself as context. In \cite{vaswani2017attention} a third set of \ti{value} vectors is introduced, resulting in three sequences termed \ti{query}, \ti{key} and \ti{value} vectors, in analogy to memory lookups.

To produce these vectors, the initial input sequence is transformed by three different learned weight matrices, namely $W^{(q)}, W^{(k)} \in \R^{d \times d_k}$ and $W^{(v)} \in \R^{d \times d_v}$.

\begin{align}
    Q & = X W^{(q)} \\
    K & = X W^{(k)} \\
    V & = X W^{(v)}
\end{align}
Then, using the obtained query and key vectors $Q$ and $K$, a matrix of attention scores is computed and matrix multiplied with $V$:

\begin{equation}
    \text{SelfAttention}(Q, K, V) = \text{softmax}\bigg(\frac{Q K^\top}{\sqrt{d_k}}\bigg) V
\end{equation}

As a consequence, each vector in the resulting sequence becomes an attention weighted sum over the vectors in $V$. Note the scaling by $\sqrt{d_k}$ which is supposed to prevent oversaturation of the softmax function, due to large dot-products.

From the memory lookup perspective: Query vectors $Q$ are matched with key vectors $K$, in order to produce compatibility scores. These scores are then used to retrieve value vectors from $V$ via soft-lookup.

\subsubsection{Multi-Head Attention}
Self-Attention can further be extended to multi-head attention by running multiple self-attention layers in parallel, then concatenating and projecting their outputs:
\begin{equation}
    \tx{MultiHeadAttention}(Q, K, V) = \bigg[ \bigparallel_{i=1}^H \tx{SelfAttention}_i(Q,K,V) \bigg] W\lay o
\end{equation}

Where $H$ is the number of attention-heads, $||$ denotes concatenation and $W\lay o \in \R^{d_v H \times d}$ is a learned matrix for projecting back to the model's original hidden dimension.

Note that in the default case, each attention layer has its own weight matrices. However,  we omit layer indices to keep the notation more simple.

\subsection{Point-wise Feed-forward}
The point-wise feed-forward component is a 2-layer MLP that is applied to each position along the sequence dimension, meaning weight parameters are shared across all positions. It follows the following architecture:
\begin{equation}
    \tx{FFN}(x) = \tx{ReLU}(x W^{(0)} + b^{(0)}) W^{(1)} + b^{(1)}
\end{equation}

\subsection{Residual Connection}
After applying a layer or block of layers in a DNN, if we add the inputs back to its outputs, it is called a residual connection or skip connection:

\begin{equation}
    \tx{Residual}(x) = f^{(k)}(x) + x
\end{equation}
Where $f^{(k)}(x)$ is a layer at depth $k$.

The most prominent motivation for residual connections is that they facilitate the training of deeper neural networks. A common issue with deep neural networks is the vanishing gradient problem. As computing gradients through backpropagation relies on a series of multiplications of potentially small values (Section~\ref{sec:backprop}), gradients tend to become smaller the further we propagate back. This makes training very deep networks harder, as early layers might receive little to no updates.

Since $\tx{residual}'(x) = f'(x) + 1$, the gradient of the residual connection will be $>1$, even if the gradient is $<1$, alleviating the effect of vanishing gradients. It can also be interpreted as preserving more of the initial input information throughout the network, treating the DNN layers as an addition to the identity function, instead of a full transformation of the input.

% vanishing gradients
% identity function idk read up
% improve gradient flow for deep models
\subsection{Layer Normalization}
Layer Normalization \cite{ba2016layer} is another technique for training deeper neural networks. When training machine learning models on numerical features, it is common practice to normalize inputs, e.g. such that their distribution is centered at $0$ and has a standard deviation of $1$. This way, there's less variance across features, resulting in more stable training and hence improving convergence.

However, since DNNs pass features through multiple layers, there's no guarantee that hidden representations will maintain a reasonable scale, meaning each layer might have to adapt to a new distribution \cite{DBLP:journals/corr/IoffeS15}. Layer Normalization tackles this problem by computing mean $\mu^{(k)}$ and standard deviation $\sigma^{(k)}$ over the feature dimension of each hidden layer $k$:

\begin{equation}
    \mu\lay k = \frac{1}{D} \sum_{i=1}^D z_i\lay k
\end{equation}

\begin{equation}
    \sigma^{(k)} = \sqrt{\frac{1}{D} \sum_{i=1}^D (z_i^{(k)} - \mu^{(k)})^2}
\end{equation}

Here $z_i^{(k)}$ denotes the $i$-th output of layer $k$ with hidden dimension $D$, before applying an activation function.\\
These layer statistics are then used to normalize the hidden layer representation $z^{(k)}$:

\begin{equation}
    \tx{LayerNorm}(z\lay k) = \gamma \lay k \circ \frac{z\lay k - \mu\lay k}{\sigma\lay k} + \beta\lay k
\end{equation}



Where $\gamma^{(k)}$ and $\beta^{(k)}$ are learned parameter vectors for layer $k$ and $\circ$ denotes the element-wise product. Further, $\gamma^{(k)}$ and $\beta^{(k)}$ are additional learnable parameters for adjusting scale and shift of the normalized distribution if required.

\subsection{The Full Transformer Block}

The full transformer block can be described with the following equations:

\begin{align}
    A              & = \tx{MultiHeadAttention}(X, X, X) \\
    Z              & = \tx{LayerNorm}(A + X)            \\
    \tx{TBlock}(X) & = \tx{LayerNorm}(\tx{FFN}(Z) + Z)
\end{align}

It consists of a multi-head attention layer and a point-wise fully connected layer, each followed by residual connection and layer normalization.


\subsection{BERT Pre-Training}

\section{Probing}

\section{MTL}
\chapter{Previous Work}
\label{chap:prev}
two part thesis, these are relevant \dots

tenney, hewitt all the good probing
effects of finetuning https://arxiv.org/abs/2004.14448
the idf paper
probing different berts
bertnesia

\section{NLP + Neural Ranking}
\section{Probing}
\section{Multitask Learning}
\chapter{Datasets}
\label{chap:datasets}

\section{TREC 2019 - Deep Learning Track}
\label{sec:trec2019}
The TREC 2019 deep learning track focuses on studying text retrieval on large-scale data \cite{DBLP:journals/corr/abs-2003-07820}. It provides two datasets, one for passage retrieval and one for document retrieval. The datasets are based on MS MARCO \cite{DBLP:journals/corr/NguyenRSGTMD16} which consists of $\sim 1$mio real world user queries from the Bing search engine and a corpus of $\sim 8.8$mio passages. For this thesis we focus on the passage retrieval dataset (TREC2019) if not stated otherwise. Further, we will refer to passages from this dataset as documents, to be consistent with the common information retrieval terminology.

\begin{table}[h]
    \centering
    \begin{tabular}{c|ccc}
        \hline
        \tf{Dataset} & \tf{Train} & \tf{Validation} & \tf{Test} \\ \hline
        Passage      & 502,939    & 55,578          & 200       \\ \hline
        Document     & 367,013    & 5,193           & 200       \\ \hline
    \end{tabular}
    \caption{Number of queries for each dataset split in the two TREC 2019 datasets.}
\end{table}

\begin{table}[h]
    \centering

    \begin{tabular}{c|c}
        \hline
        Passage   & Document  \\ \hline
        8,841,823 & 3,213,835 \\ \hline
    \end{tabular}
    \caption{Corpus size for each TREC dataset.}
\end{table}

While with TREC2019, two types of tasks are provided, namely full ranking and re-ranking, we will only perform the re-ranking task. This means, given a pool of $1000$ documents for each query, we need to provide an ordering, such that relevant documents are placed at the top. On average, a query has $\sim 1.1$ relevant documents in its pool which were marked as relevant by human annotators. Note that each annotator only had access to $\sim 10$ passages during annotation, meaning a pool is likely to contain false negatives, i.e. relevant passages that are not marked as such.

\section{Probing Dataset Generation}
\label{sec:dataset_gen}
For all of our probing tasks, we automatically generate datasets from the TREC2019 passage-level test set. To achieve this, we sample $60$k query-document pairs from the test set of which $40$k are used as training set and $10$k as validation and test set, respectively. We then use existing tools to extract the properties that we are interested in and use them to label the data. In the following we explain our usage of tools for each dataset.

\subsection{BM25 Prediction}
To generate BM25 scores we leverage the Elasticsearch BM25 implementation\footnote{\url{https://www.elastic.co/de/elasticsearch/}}. We first index the full test dataset, then compute bm25 for each query pool, and uniformly sample the $60$k query-document pairs.

\subsection{Named Entity Recognition}
For identifying named entities we use spacy's \cite{spacy2} named entity recognition module. It is capable of detecting and assign one of $18$ different types of entities. We only include pairs that contain at least one entity.


\subsection{Semantic Similarity}
To compute semantic similarity between query and document we first embed all words in the GloVe \cite{pennington2014glove} vector space. We then compute the average embedding for each query and document over the sequence dimension and compute their cosine similarity (\autoref{eq:sem_sim}).

\subsection{Coreference Resolution}
To detect and score entity pairs, we use the neuralcoref pipeline extension for spacy\footnote{\url{https://github.com/huggingface/neuralcoref}}. It consists of a rule-based mentions-detection module, followed by a feed-forward neural network which produces a binary coreference score for each detected pair.

\subsection{Fact Checking}
Fact checking is the only dataset that we don't sample from TREC2019. Instead, we leverage the existing labels of the FEVER fact check dataset \cite{thorne-etal-2018-fever} and sample claim-evidence pairs from the train set.
\chapter{Approach}
\label{chap:approach}

\section{Methodology}

\section{Experimental Setup}

\section{Evaluation Measures}
\subsection{MDL}
\subsection{Compression}
\subsection{F1}
\subsection{Accuracy}
\subsection{Ranking}
\subsubsection{MAP}
\subsubsection{MRR}
\subsubsection{NDCG}
\subsubsection{Precision}


\chapter{Probing Results}
\label{chap:results}
In this section we will analyze and discuss the results from the probing experiments. Since the overall distribution of properties across layers follows a similar pattern across models, we will first focus on the distribution in general, then discuss the effects of fine-tuning in the following section.

\section{Distribution of Ranking Properties}
Firstly, a general trend we can observe across tasks, is an increase in both compression and accuracy over the first couple of layers up to layer 4-6, suggesting that ranking concepts arise mostly in the mid-range of layers. Then, after a peak at some mid to upper level layer, we observe a constant decrease until the last layer. It is notable that accuracy seems to exhibit a less stable curve, with sudden peaks and drops in between adjacent layers, especially in the case of coreference \autoref{fig:coref} and semantic similarity \autoref{fig:sem_sim}. This observation coincides with \cite{voita-titov-2020-information} findings, that MDL and as a consequence compression, are a more reliable measure for probing than accuracy.

For the most part, based on compression, ranking properties can be decoded more efficiently from our trained models than from the random baseline, meaning the probe model is not solely adapting to the task, but instead leveraging information present in the pre-trained representations.

We can further observe that the difference in compression between early layers and the peak value varies depending on the task. This Indicates that some properties are more uniformly distributed across layers, while others are more concentrated at a particular layer. For example, decoding named entities results in similar compression scores from layer 1-11, while semantic similarity shows a distinct peak at layer 4.

To better understand this behavior, we can have a look at the row-normalized heatmap in \autoref{fig:heatmap_comp_passage} of the \ti{bert-msm}. We can see that coreference and fact checking strongly center around layers 5-9, with fact checking showing a slightly wider spread. While bm25 exhibits a similar pattern, it is less focused around a particular layer, but instead more evenly distributed from layer 4-7. Semantic similarity and NER on the other hand look almost identical with a rather flat distribution that appears to be centered around layer 4.

When considering that coreference and fact checking are both tasks that require higher level semantics, based on the observation in \cite{tenney-etal-2019-bert}, it makes sense that both distributions are leaning more towards mid to upper layers. On the other hand, semantic similarity is a property that is already captured by non-contextual word embeddings. We hypothesize that as a fundamental concept, that higher level tasks build on, it's a property of the embedding space that needs to be preserved throughout the whole model. 


ner like semantic sim non intuitive, however grouping of concepts in vector space, eg word 2 vec visualization

bm25 cite idf paper, co-occurence information - document level, not only structural?


\todo{compute centers of gravity?}

heatmap: per task distribution consequences

absolute for comparison across tasks

coref and ner already well encoded, sem-seim surprisingly not as easy. due to finetuning?

\section{Effects of Fine-tuning}
passage more than doc and base
drop in last layer
increase not in early but over mid layers, cite tenney finetune


\begin{figure}%[hb!]
    \centering
    \makebox[\textwidth][c]{\includegraphics[width=1.25\textwidth]{gfx/probing/bm25}}
    \caption{\dots}
\end{figure}

\begin{figure}
    \label{fig:sem_sim}
    \centering
    \makebox[\textwidth][c]{\includegraphics[width=1.25\textwidth]{gfx/probing/sem_sim}}
    \caption{\dots }
\end{figure}

\begin{figure}
    \label{fig:coref}
    \centering
    \makebox[\textwidth][c]{\includegraphics[width=1.25\textwidth]{gfx/probing/coref}}
    \caption{\dots}
\end{figure}

\begin{figure}
    \centering
    \makebox[\textwidth][c]{\includegraphics[width=1.25\textwidth]{gfx/probing/ner}}
    \caption{\dots}
\end{figure}

\begin{figure}
    \centering
    \makebox[\textwidth][c]{\includegraphics[width=1.25\textwidth]{gfx/probing/fact_check}}
    \caption{\dots}
\end{figure}



\begin{figure}
    \label{fig:heatmap_comp_passage}
    \includegraphics[width=\textwidth]{gfx/probing/heatmap_compression_passage}
    \caption{\ti{bert-msm-passage}: Compression as a function of task and layer, row-normalized. Darker colors represent higher values.}
\end{figure}

\begin{figure}
    \label{fig:heatmap_comp_base}
    \includegraphics[width=\textwidth]{gfx/probing/heatmap_compression_base}
    \caption{\ti{bert-base-uncased}: Compression as a function of task and layer, row-normalized. Darker colors represent higher values.}
\end{figure}


\begin{figure}
    \centering
    \includegraphics[width=\textwidth]{gfx/probing/abs_heatmap_compression_passage}
    \caption{\ti{bert-msm-passage}: Compression as a function of task and layer, absolute values.}
\end{figure}

\begin{figure}
    \centering
    \includegraphics[width=\textwidth]{gfx/probing/abs_heatmap_compression_base}
    \caption{\ti{bert-base-uncased}: Compression as a function of task and layer, absolute values.}
\end{figure}


\chapter{Informed Multi-Task Learning}
\todo{explain motivation}
\section{Model Architecture}
\section{Experimental Setup}
\section{MTL Results}
\section{Ablation}
% investigate if valid augmentation technnique in low resource scenario
\chapter{Conclusion}
\label{chap:conclusion}
Throughout this thesis we've explored to what extent different ranking abilities are encoded by a BERT model's word representations. Through probing, we found that different layers are better at encoding certain properties. Whereas lower level layers were better at capturing task knowledge that is helpful for simple, match based query-document relations (SEM, BM25, NER), we found mid- to high-level layers to hold more complex semantic information (COREF, FACT CHECK).

Furthermore, we've investigated how fine-tuning affects the distribution of properties, by probing a version of the model that has been adapted for ranking. We were able to observe a shift in distribution towards mid- and upper-layers and a consistent loss of information in the last layer, regardless of the task. In addition, by comparing absolute compression values between tasks, we found that the overall presence of property information increased through fine-tuning, suggesting that these properties are indeed relevant for learning a ranking model. Interestingly, this effect was especially pronounced with the COREF property.

Then, based on our findings, we've designed a new multi-task learning setup to aid fine-tuning BERT for ranking and found that infusing task information at specific layers can indeed help to learn a better ranking model and that the choice of layer does matter.

For this, we've investigated two ranking objectives, pointwise and pairwise, with different additional task combinations. Through this, we found that for the pointwise objective, providing a single additional task would result in increased ranking performance. However, when increasing the number of tasks the performance would drop. Surprisingly, we could not observe this decrease in performance when using the pairwise ranking objective. Instead, the overall performance did still increase.

\section{Limitations}
\label{sec:limitations}
Firstly, there is a general limitation that arises when using the probing framework. Even though a probe can help us estimate whether a certain property is extractable from a model's representations, we can not conclude that the model itself actually uses that information during inference. While we can gather more evidence on whether a property is important for a downstream task, by comparing to a fine-tuned model, this is still not a guarantee.

Another problem is the quality of automatically generated task data. Because some of our tools used for label generation also rely on learned models, this will certainly result in some wrong predictions, meaning our task labels are to a certain extent noisy, ultimately causing less reliable probing results.

Regarding our MTL experiments, while we could observe a general improvement in ranking performance across the board, this improvement was rather marginal. Whether this is due to the model being too simple, lacking quality in additional task data, or simply because the model already has access to most of the infused information, has yet to be explored.

\section{Future Work}
\label{sec:future}
Considering the limitations of this thesis, we suggest the following directions for future work:
\begin{itemize}
    \item There are still more ranking properties that one might think of. Probing BERT for additional ranking properties can provide further insights on neural ranking.
    \item By using more sophisticated approaches, the automatic label extraction process might be improved, in order to produce higher quality probing datasets.
    \item In this thesis we've only used the TREC2019 dataset. Leveraging additional datasets might give us more general insights and also help with MTL learning.
    \item Since we've used a fairly simple MTL architecture, it might be beneficial to extend our approach with more advanced architectural designs.
\end{itemize}
Whereas this is just a small selection of potential improvements building on this thesis, there is still much more work to be done in understanding the capabilities of neural ranking models.


\chapter*{Plagiarism Statement}
\addcontentsline{toc}{chapter}{Plagiarism Statement}
\vfill
\mbox{} \\
{\large I hereby confirm that this thesis is my own work and that I have documented all sources used. I have not submitted this thesis for another class or module (or any
other means to obtain credit) before.}
\newline
\mbox{} \\
Hannover, \handindate \\
\vspace{4cm}
\hrule
\vspace{0.5cm}
$\qquad$(Fabian Beringer)

\listoffigures
\addcontentsline{toc}{chapter}{\listfigurename}
\listoftables
\addcontentsline{toc}{chapter}{\listtablename}
\clearpage
\phantomsection
\addcontentsline{toc}{chapter}{Bibliography}
\bibliography{bibliography.bib}
\end{document}